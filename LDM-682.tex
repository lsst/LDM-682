% vim: tw=0:wrap:linebreak
\documentclass[DM,toc,lsstdraft]{lsstdoc}

\usepackage{comment}
\usepackage{datetime}
\usepackage{microtype}

\newcommand{\microarcsec}{$\mu$as\xspace}
\interfootnotelinepenalty=10000


\setcounter{secnumdepth}{3}

%%%%%%%%%%%%%%%%%%%%%
% Introduce mechanism to turn on and off various annotations (\XXX command,
% \begin{notes}, and anything else bracketed by \ifannotated ... \fi)
%
\newif\ifannotated
\annotatedtrue
\annotatedfalse	% uncomment this to hide all annotations (comments, notes, etc)

\ifannotated
	% leave things as-is
\else
	% hide all \XXX commands
	\renewcommand{\XXX}{}

	% hide all \begin{note}...\end{note} text
	\renewenvironment{note}[1][Note]
	{}
	{}
\fi
%%%%%%%%%%%%%%%%%%%%%

\title{Call for Letters of Intent for Community Alert Brokers}
\author{
Eric~Bellm,
Robert~Blum,
Melissa~Graham,
Leanne~Guy,
\v{Z}eljko~Ivezi\'{c},
William~O'Mullane,
and John~Swinbank
\emph{for the LSST Project}
}

\input{meta}
\setDocRef{\lsstDocType-\lsstDocNum}
\setDocCurator{E.~Bellm}
\setDocDate{\vcsdate}
\setDocUpstreamVersion{\vcsrevision}
\setDocUpstreamLocation{\url{https://github.com/lsst/LDM-682}}

\setDocAbstract{%
	abstract
}

%
%   Revision history
%
% OLDEST FIRST: VERSION, DATE, DESCRIPTION, OWNER NAME
\setDocChangeRecord{%
\addtohist{}{2019-01-17}{Initial draft version}{Eric Bellm}
}


\begin{document}


\setDocAbstract{%
A major product of the nightly processing of LSST images is a world-public stream of alerts from transient, variable, and moving sources. Science users may access these alerts through third-party community brokers, which will receive the LSST alerts, add scientific value, and redistribute them to the scientific community.
 
This document is a call for Letters of Intent (LOIs) to propose a community broker, as described in ``Plans and Policies for LSST Alert Distribution'' \citedsp{LDM-612}.
}

\maketitle

\section{Process}

The broker selection process will have two stages:
an initial open call for LOIs from all interested parties (described in this document), and a subsequent full proposal call solicited from invited LOI writers.
This two-stage process ensures those writing full proposals have a reasonable chance at selection and provides an opportunity for early feedback.
An invitation to submit a full proposal may also be valuable to proposers in obtaining relevant funding.
No financial support is available from the LSST Project.


\section{LOI Guidelines}

\subsection{Who Can Submit}

Individuals at any institution worldwide (academic, nonprofit, or commercial) may submit an LOI.
Submission is open to all, regardless of whether or not an individual holds LSST data rights.


\subsection{Page Limits}

LOIs should be no more than four pages, with standard margins and font sizes.
A \LaTeX\ template is available at \url{https://github.com/lsst/LDM-682/blob/master/LOI_submission_template.tex}.

\subsection{Due Date}

For full consideration, LOIs should be submitted by May 15, 2019.
Submission of an LOI is required in order to be invited to submit a full proposal.

\subsection{Content}


Proposers are encouraged to read \citeds{LDM-612}: Plans and Policies for LSST Alert Distribution, in preparing their proposal. 
The LOI should address at a high level the following items:

\begin{itemize}
	\item Scientific goals of the proposed broker system
	\item Anticipated data products \& services
	\item Envisioned technical implementation
	\item Management Plan
	\item Description of proposing team
	\item Relevant experience 
\end{itemize}

The \LaTeX\ template provides a handful of prompts for each section.

\subsection{Submission Instructions}

Please submit LOIs as PDF documents to \url{lsst-community-brokers@lists.lsst.org}.

\section{Broker Workshop}


LOI submitters are invited to attend a workshop in Seattle the week of June 17, 2019.
This technical workshop will bring together LSST Project personnel, representatives of the LSST Science Collaborations, and proposers of community brokers.  
Topics of discussion will include LSST-broker interfaces, architecture and technology choices, the broker selection process, policy issues, science use cases and expectations, and development progress from precursor surveys.
More information will be forthcoming soon.

\section{Selection Process and Timeline}

It is understood that proposed brokers may be in formative stages at the time of this call.
Please refer to \citeds{LDM-612} \S4.5--4.6 for evaluation criteria and procedures.

We anticipate notifying invited LOI writers in August 2019.
As described in \citeds{LDM-612} we expect to issue the call for full proposals to invited teams in Q4 2019, with an anticipated due date in Q2 2020.

%\clearpage
%\section{Acronyms}
%\input{acronyms}

\bibliography{lsst,refs_ads,refs,local}


\end{document}
